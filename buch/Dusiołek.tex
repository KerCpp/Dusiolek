\documentclass[12pt]{book}
\usepackage[polish]{babel}
\usepackage[T1]{fontenc}
\usepackage[utf8]{inputenc}
\usepackage{indentfirst}
\title{\textbf{Tytuł Magdy,} \\\textit{czyli rzecz o marach i dziwach strasznych}}
\author{$MB^2$}

\begin{document}
\maketitle
Wiataj Brambeł!\\\par
Zastanawiasz się pewnie o co chodzi? Od dziś zaczynasz nową przygodę. Koniec z lenieniem się! Piszemy das Buch, libro, gr\=amata! Teraz przedstawię Ci zasady:
\begin{itemize}
\item[1.]Każdy pisze na raz tylko jeden akapit, ten akapit może rozpoczynać nowy rozdział.
\item[2.]Akapit musi tworzyć logiczną całość z akapitami go poprzedzającymi.
\item[3.]Czas na dodanie nowego akapitu wynosi dwa dni, ewentualnie trzy. Przekroczenie terminu będzie karane zaproszeniem na czaj.
\item[4.]Jedno zdanie to nie akapit.
\item[5.]Nie wolno się śmiać z tego, co napisze Michał Bożek, błędy można poprawiać.
\item[6.]Ziemniaczek!
\end{itemize}
\chapter{Licho w szafie}
Od dawna babki, matki i ojcowie wiedzą, że niegrzeczne dziecko ściąga na rodzinę nieszczęścia. Gniewomir nie mógł się nie zgodzić z tą mądrością, ojciec wpajał mu ją postronkiem sukcesywnie przez piętnaście lat. Jako wioskowy kowal czynił to naprawdę dobrze. Jego dwaj wnukowie również zasmakowali ojcowskiej miłości. Jedna tylko Żywia nigdy nie uczestniczyła w procesie edukacji. Ojciec uważał ją za zbyt delikatną. Dziś, przygotowując się do obchodów święta Jarowita, Gniewomir nie mógł się pozbyć myśli, że znów będzie żałował swojej pobłażliwości. \par
Poranek upłynął zgodnie z planem. Borzuj i Derwan pracowicie wykonywali przydzielone im zadania, nie zakłócając spokoju ojca ani jednym niepotrzebnym dźwiękiem, co bliźniakom zdarzało się niezwykle rzadko; widocznie wczorajsza lekcja wciąż rozbrzmiewała boleśnie w ich pamięci. Jabłka i bułeczki dla późnojesiennych\footnote{w sumie to jest około 15 kwietnia :P} chochlików, które miały przybyć późną nocą do wioski w imieniu Jarowita, zostały rozłożone pod każdym oknem, podobnie jak kostki cukru, przeznaczone dla towarzyszących im wędrownych gnomów. Witki wierzbowe do przeganiania bobików czekały obok drzwi do chaty, w razie gdyby złośliwe stworki nieopatrznie przybłąkały się do wioski z rozciągającego się na wschodzie wielkiego Boru Boginek. \par
-- Hej! -- krzyknął Derwan. Żywia idąca nieopodal odwróciła się lekko. Przybrawszy poważną minę, kontynuował: -- Dziewuchy siedzą dziś w chałupie.\\
Żywia spojrzała tylko na niego. Pewność siebie obecna w głosie jej młodszego brata ją bawiła. Nie chciała się jednak z nim sprzeczać. Mimo że był młodszy, to siłą przewyższał ją wielokrotnie. Nieraz zdarzyło się, że ukrywała siniaki po bójce z bratem. Mimo to odpowiedziała rezolutnie:\\
\indent -- A Ty się nie boisz, Derwan? Słyszałam, że w zeszłym roku dopadł Cię czort z lasu. \\ 
Borzuj spojrzał ostrzegawczo na czerwieniącego się brata. Najroślejszy z rodzeństwa był też i najposłuszniejszy przykazaniom Gniewomira. Mimo że jeszcze nie przeszedł obrzędu inicjacji, to często był traktowany jako pełnoprawny członek społeczności. Wywołane tym pewność siebie i duma były solą w oku reszty rodzeństwa.\par

W końcu nadeszło jednak to święto Jarowita, którego Borzuj tak oczekiwał: ostatniego dnia poprzedniego roku skończył dziewiętnaście lat, więc będzie mógł wreszcie wziąć udział w tajemniczych rytuałach. Nadchodzącego wieczora udowodni ojcu i rodzeństwu, że zasługuje, by traktować go jak dorosłego. \\
-- Nie powinnaś wspominać imion tych stworzeń -- upomniał siostrę, przybierając surową minę. -- No i dobrze wiesz, że nic takiego nie miało miejsca.
Żywia prychnęła z dezaprobatą, nie omieszkając też rzucić mu groźnego spojrzenia.\\
-- Chwilami zachowujesz się jak ojciec. A ja mogę robić to, na co mam ochotę. Jeszcze nie możesz mi niczego zabronić.\\
-- A jeśli dzisiaj zawiedziesz... -- Derwan odezwał się i od razu tego pożałował, gdy zobaczył minę brata. -- Oczywiście, że nie zawiedziesz. Ale ostatnimi czasy to się zdarza niepokojąco często, słyszałem, jak ojciec rozmawiał na ten temat ze stryjem...\\
Niepowodzenie w obrzędzie inicjacji oznaczało wygnanie z wioski, pozostawienie za sobą rodziny i dotychczasowego życia. Na tę myśl Borzuj czuł narastający niepokój, pochłaniający jego dobry humor niczym czarna otchłań, z której nie ma powrotu. Gdzie by poszedł w razie klęski? Co by go spotkało na drodze, którą by podjął?\\
-- Nie opowiadaj głupot! Borzuj świetnie sobie poradzi. Jeśli chłopak Braturada był w stanie to przejść, to nasz brat na pewno da sobie radę.\\
Borzuj uśmiechnął się słabo. Mogło mu się to uroić, ale chyba wyczuł niepewność w głosie siostry. Jest zawsze pewna siebie, a głos zwykle jej nie drży, gdy wypowiada opinie.\\
-- Bardzo ci dziękuję za to porównanie. Nie ma nikogo bardziej leniwego i niezdarnego w naszej wiosce od niego.\par



$>>$
Ofiarowałem Ci footnote w ramach ciekawostki no i ejjj nie mamy tytułu
\end{document}
